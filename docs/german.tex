\documentclass[a4paper,11pt]{amsart}
\usepackage[ngerman]{babel}
\usepackage[utf8]{inputenc}
\usepackage{geometry}
\usepackage{booktabs}
\usepackage{graphicx}
\usepackage{listings}

\title[Generieren schematischer Verkehrsnetzkarten]{Automatisches Generieren schematischer Verkehrsnetzkarten als Problem der ganzzahligen Optimierung (MIP)}
\author{Julius Tens, Dirk Schumacher}
\email{tensjuli@math.hu-berlin.de}
% \date{1. August 2018}

% todo: edge ordering / regions

\begin{document}
\maketitle

\section*{Problem}

\noindent Gegeben seien ein \textit{einfacher}, \textit{ungerichteter} Graph $G(V,E,L)$ bestehend aus Knoten $V$, Kanten $E$ und Linien $L$, sowie eine \textit{geradlinig planare} Einbettung $P$ dieses Graphen in die Ebene. Zu jedem Knoten $v_n$ sei also ein kartesisches Koordinatenpaar $(x_n | y_n) \in P$ bekannt (in der Regel über eine geeignete Projektion aus Geokoordinaten ermittelt), wobei es keine Schnittpunkte zwischen nicht benachbarten Kanten geben darf.
\\\\
Gesucht sind nun weitere \textit{geradlinige} Einbettungen $P'$ desselben Graphen $G(V,E,L)$ in die Ebene, die die folgenden Bedingungen erfüllen \textit{(hard constraints)}:
\bigskip

\begin{description}
\item[Planarität] $P'$ ist eine planare Einbettung von $G$ in die euklidische Ebene.
\bigskip
\item[Oktilinearität] Alle Kanten müssen Strecken sein, die parallel oder senkrecht zur x-Achse oder zur Identität verlaufen.
\bigskip
% \item[Bewahrung der Regionen] $P'$ enthält die gleichen Regionen (\textit{Faces}) wie $P$, d.h. die „Sortierung der Kanten“ an jedem Knoten muss gleich bleiben.
% \bigskip
\item[Mindestlängen] Jede Kante muss eine Länge $l_e >= l_{min}$ (in Tschebyschow-Norm) haben.
\bigskip
\end{description}

% todo: soft constraints als liste
\noindent Es wird insbesondere eine Einbettung $P'$ gesucht, die möglichst kurze Kantenlängen sowie wenige, bestenfalls stumpfwinklige „Knicke“ pro Linie hat \textit{(soft constraints)}.
\bigskip
\bigskip
\hrule
\hrule
\section*{Ansatz}

\noindent Wie bereits von \textsc{Nöllenburg et al.} gezeigt, lässt sich das Problem als Problem der ganzzahligen Optimierung \textit{(MIP)} formulieren.
\bigskip
\bigskip
\hrule
\hrule
\section*{Lemmata}

\noindent Bei der Formulierung des Modells wird auf die folgenden Hilfskonstrukte zurückgegriffen, mit denen die meisten Leserinnen und Leser jedoch vertraut sein sollten.

\subsection*{Nicht-Gleichung} $x \ne y$ kann mit Hilfe einer zusätzlichen binären Variable $b$ linearisiert werden:
\begin{align*}
x - y &\le - \varepsilon + M b\\
x - y &\ge \varepsilon - (1-b)\cdot M
\end{align*}
wobei $\left|x-y\right| \gg \varepsilon > 0$, $M \gg x$ und $M \gg y$.
\bigskip


\subsection*{Countinous-Binary-Produkt}$x = A \cdot b$ mit $M \gg A \in \mathbb{Q_+}, b \in \{0, 1\}$ kann folgendermaßen linearisiert werden:
\begin{align*}
x &\le M b\\
x &\le A\\
x &\ge A - (1-b) \cdot M\\
x &\ge 0\\
\end{align*}
\hrule
\hrule

\section*{Formulierung}

\noindent Es sind $x_n, y_n$ die Koordinaten des Knoten $n$ in der gesuchten Einbettung $P'$, sowie $l_e$ die Länge (in Tschebyschow-Norm) der Kante $e$:
\bigskip
\begin{gather*}
x_n, y_n \in \mathbb{Q}\\
l_e \in \mathbb{N} \\
l_{min} \le l_e \le l_{max}
\end{gather*}
\bigskip

\noindent Des Weiteren wird für jede Kante $e$ einer ihrer Knoten als Startpunkt $S_e$ (\textit{source}) festgelegt, der zweite Knoten ist dann Zielpunkt $T_e$ (\textit{target}).\\

\noindent Die \textit{hard constriants} lassen sich nun unter Zuhilfenahme jeweils aufgeführter weiterer Variablen wie folgt formulieren:
\bigskip
\hrule

\subsection*{Oktilinearität und Mindestlängen}
Für jede Kante $e$ werden binäre Variablen $a_e, b_e, c_e, d_e$ eingeführt, für die gilt:
\bigskip
\begin{gather*}
a_e, b_e, c_e, d_e \in \{0, 1\}\\
x_{T_e} - x_{S_e} = (a_e + b_e) \cdot l_e\\
a_e + b_e \le 1\\
y_{T_e} - y_{S_e} = (c_e + d_e) \cdot l_e\\
c_e + d_e \le 1
\end{gather*}

\bigskip

\noindent Die binären Variablen $a_e, b_e, c_e, d_e$ werden zudem durch drei weitere Gleichungen so eingeschränkt, dass die Orientierung einer Kante in $P'$ nur eine der beiden der in der Originaleinbettung $P$ nächstliegenden oktilinearen Orientierungen sein kann.\\\\
Ein Beispiel: Die Kante mit dem Richtungsvektor $(1, 2)^T$ in $P$ kann in $P'$ nur den Richtungsvektor $(1, 1)^T$ ($a = 1$, $b = 0$, $c = 1$, $d = 0$) oder $(0, 1)^T$ ($a = 0$, $b = 0$, $c = 1$, $d = 0$) haben. Daraus folgen in diesem Fall die folgenden Gleichungen:
\bigskip
\begin{gather*}
b = 0\\
c = 1\\
d = 0
\end{gather*}

\bigskip
\noindent Mit $a, b, c, d$ sowie den Variablen für die Produkte aus Binary und Continuous werden pro Kante also insgesamt 4 binäre und 2 kontinuierliche Variablen hinzugefügt, sowie 15 (Un-) Gleichungen.
\bigskip
\hrule

\subsection*{Planarität}

Die gesuchte Einbettung $P'$ ist planar, wenn sich nicht-benachbarte Kanten in keinem- und benachbarte Kanten in genau einem Punkt berühren bzw. schneiden.

\subsubsection*{Benachbarte Kanten} Für jedes Paar anliegender (benachbarter) Kanten $p, q \in E$ werden die folgenden Gleichungen aufgestellt:
\bigskip
\begin{gather*}
\left\{\begin{array}{lr}
    3 (a_p + b_p) + (c_p + d_p) \ne 3 (a_q + b_q) + (c_q + d_q), \text{ &falls } T_p = S_q \lor T_q = S_p \\
    3 (a_p + b_p) + (c_p + d_p) \ne 3 (b_q - a_q) + (d_q - c_q), \text{ &andernfalls}
\end{array}\right\}
\end{gather*}

\bigskip
\noindent Für jede Kante mit $k$ benachbarten Kanten werden also $2k$ Ungleichungen und $k$ binäre Variablen hinzugefügt.

\subsubsection*{Nicht-benachbarte Kanten}

Um sicherzustellen, dass zwei nicht-benachbarte Kanten $p$ und $q$ keine Schnitt- oder Berührpunkte haben, müssen einige Vorüberlegungen angestellt werden:\\\\
Sei $G$ die Menge aller Geraden in der euklidischen Ebene, die \textit{zwischen} $p$ und $q$ verlaufen, ohne diese zu berühren oder zu schneiden. Zudem sei $O$ die Menge aller oktilinearen Geraden in $G$. Dann ist die Menge $M$ wie folgt definiert:
\bigskip
\begin{gather*}
M = \left\{\begin{array}{lr}
    O, \text{ &falls } O \ne \emptyset \\
    G, \text{ &andernfalls}
\end{array}\right\}
\end{gather*}

\bigskip
\noindent Aus der Planarität der Einbettung $P$ folgt im Übrigen $M \ne \emptyset$.\\\\
Wir definieren des Weiteren den Abstand zweier Kanten \textit{bezüglich einer Geraden $g$} als euklidische Länge der kürzestmöglichen, zu $g$ orthogonalen Strecke, die beide Kanten berührt. \\\\
Gesucht wird nun eine Gerade $m \in M$ mit größtmöglichem Abstand zwischen $p$ und $q$ \textit{bezüglich $m$}.\\\\
Hat $m$ die Geradengleichung $ax + by + c = 0$ und ist o.B.d.A. $ax_S_p + by_S_p < ax_S_q + by_S_q$, stellen die folgenden 4 Gleichungen sicher, dass sich die nicht benachbarten Kanten $p$ und $q$ weder berühren noch schneiden (mit Mindestabstand $d_{min}$):
\bigskip
\begin{gather*}
ax_S_p + by_S_p + d_{min} < ax_S_q + by_S_q\\
ax_S_p + by_S_p + d_{min} < ax_T_q + by_T_q\\
ax_T_p + by_T_p + d_{min} < ax_S_q + by_S_q\\
ax_T_p + by_T_p + d_{min} < ax_T_q + by_T_q
\end{gather*}

\bigskip
\noindent Für jede Kante mit $k$ nicht-benachbarten Kanten werden also $4k$ Ungleichungen, aber keine weiteren Variablen hinzugefügt.
\bigskip
\hrule
\hrule

% todo

\bigskip
\bigskip
\bigskip
\end{document}
